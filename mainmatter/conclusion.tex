% Conclusion
% These techniques are underrepresented because of a variety of reasons, one of them being that there is a lack of resources dedicated to writing for them.
% It is hoped that this exegesis will contribute to their more widespread adoption. 
% It becomes apparent that the literature surrounding these techniques still lacks comprehensive guidelines over the techniques' use, and the works that have been produced have been through trial and error.
\section{Impact and Further Research}
This exegesis will help inform other artists interested in implementing these techniques. 
Compositionally, the scope of this exegesis has been limited to the techniques appropriate for solos, and no research into how the techniques fit into ensemble works has been attempted.
The exact mechanics of the production of \hyperref[sec:subharmonics]{subharmonics} and \hyperref[sec:multiphonics]{multiphonics} are still poorly understood, and would benefit from further research.
Further research into the way the techniques react to artificial harmonics, the difference between the two nodal points for multiphonics, and the methods for producing different intervals of subharmonics is needed for a holistic understanding.
The analysis and cataloguing of the qualities of each multiphonic and subharmonic would contribute further to the ideal of idiomatic writing for the techniques.


\addcontentsline{toc}{chapter}{Conclusion}
\section{Conclusion}

It is apparent that the literature surrounding these techniques is still in its infancy, with few sources of authority due to the niche nature of the techniques.
In this exegesis, the documentation of the existing literature and findings from the implementation of the techniques establishs a baseline for treating these techniques, which others can build upon.
Through the gradual adoption of these techniques, a standardised notation will form, and further advance the acceptance of these techniques in modern literature.

