% Conclusion
These techniques are underrepresented because of a variety of reasons, one of them being that there is a lack of resources dedicated to writing for them.
It is hoped that this exegesis will contribute to their more widespread adoption. 

\section{Impact and Further Research}
This exegesis will help inform other artists interested in implementing these techniques. 
Compositionally, the scope of this exegesis has been limited to the techniques in a soloistic context, and no research into how the techniques fit into an ensemble context has been done as of the time of writing.
The exact mechanics of the production of \hyperref[sec:subharmonics]{subharmonics} and \hyperref[sec:multiphonics]{multiphonics} are still poorly understood, and further research into the way strings react to perversion of the Helmholtz motion and other non-standard ways of playing can be done.
Subharmonic intervals other than a minor second or octave have also not been explored in this exegesis.