% Conclusion
% These techniques are underrepresented because of a variety of reasons, one of them being that there is a lack of resources dedicated to writing for them.
% It is hoped that this exegesis will contribute to their more widespread adoption. 
% It becomes apparent that the literature surrounding these techniques still lacks comprehensive guidelines over the techniques' use, and the works that have been produced have been through trial and error.
\section{Impact and Further Research}
This exegesis will help inform other artists interested in implementing these techniques. 
Compositionally, the scope of this exegesis has been limited to the techniques in a soloistic context, and no research into how the techniques fit into an ensemble context has been done as of the time of writing.
The exact mechanics of the production of \hyperref[sec:subharmonics]{subharmonics} and \hyperref[sec:multiphonics]{multiphonics} are still poorly understood, and further research into the way strings react to perversion of the Helmholtz motion and other non-standard ways of playing can be done.
The analysis and cataloguing of the qualities of each multiphonic and subharmonic would contribute further to the ideal of idiomatic writing for the techniques.


\addcontentsline{toc}{chapter}{Conclusion}
\section{Conclusion}
It becomes apparent that the literature surrounding these techniques is still in its infancy, with few sources of authority due to the niche nature of the techniques.
In this exegesis, the documentation of the existing literature and findings from the implementation of the techniques has helped establish a baseline of how to treat these techniques, which others can build upon.
Through the gradual adoption of these techniques, a standardised format of notating these techniques will form, and further proliferate the acceptance of these techniques in modern literature.

