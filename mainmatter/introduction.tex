\section{Introduction}
\addcontentsline{toc}{chapter}{Introduction}


% Set page numbering to arabic the first time we commence a chapter.
% This is required to get the page numbering correct.
\pagenumbering{arabic}
\doublespace{}

I propose to explore a range of extended techniques that utilise the harmonic series and assess how they can be used in my, and other people's, creative practice. 
These techniques are multiphonics, subharmonics, and half harmonics, confined to string based instruments due to the scope of this exegesis.
For the purposes of brevity, these harmonic-based extended techniques will simply be referred to as `techniques' throughout the paper, except for when differentiation between standard techniques is needed.

While some techniques such as harmonics are well established and understood, others, such as subharmonics, are still underdeveloped in terms of both repertoire and resources available. 
The timbral potentials of these techniques are uncharted territories and collectively represent an entire sound world that remains relatively inaccessible to composers.
% TODO: Evidence for uncharted territories - https://trello.com/c/u605dDsM/15-evidence-for-uncharted-territories

% To identify where further research is required, I will conduct a review of the literature and resources that are readily available to composers to assess what techniques require further investigation and refinement. 
% By researching these techniques and the mechanics behind them, interviewing professionals, and analysing recordings made, I hope to gain a better understanding of how these techniques can be implemented in my practice. 
% As part of both the analysis of techniques and my compositional practice, I will assess not only the compositional potential, but also the practicality of techniques. 
% Reviewing the feasibility and notational aspects of the techniques will render the exegesis a practical document to reference for performance and composition.

I aim for my resulting exegesis to become a useful reference source for artists interested in learning about the mechanics, qualities, and potential implementations of these harmonic based extended techniques. 
The works that I compose accompanying the exegesis will show idiomatic treatment of the techniques and serve as references as such in the exegesis.
This exegesis makes use of hyperlinks throughout to promote its use as a reference document. 
The dissemination of the material I research will contribute to the accessibility of new sound possibilities for artists.