% \documentclass[12pt]{report}
% \usepackage[a4paper]{geometry}
% \usepackage[myheadings]{fullpage}
% \usepackage{fancyhdr}
% \usepackage{lastpage}
% \usepackage{graphicx, wrapfig, subcaption, setspace, booktabs}
% \usepackage[T1]{fontenc}
% \usepackage[font=small, labelfont=bf]{caption}
% \usepackage{fourier}
% \usepackage[protrusion=true, expansion=true]{microtype}
% \usepackage[english]{babel}
% \usepackage{sectsty}
% \usepackage{url, lipsum}


% \newcommand{\HRule}[1]{\rule{\linewidth}{#1}}
% \onehalfspacing
% \setcounter{tocdepth}{5}

% \setcounter{secnumdepth}{5}

% %-------------------------------------------------------------------------------
% % HEADER & FOOTER
% %-------------------------------------------------------------------------------
% % \pagestyle{fancy}
% % \fancyhf{}
% % \setlength\headheight{15pt}
% % \fancyhead[L]{Student ID: 1034511}
% % \fancyhead[R]{Anglia Ruskin University}
% % \fancyfoot[R]{Page \thepage\ of \pageref{LastPage}}
% %-------------------------------------------------------------------------------
% % TITLE PAGE
% %-------------------------------------------------------------------------------

% \begin{document}

% \title{ \normalsize \textsc{for solo violin}
% 		\\ [2.0cm]
% 		\HRule{0.5pt} \\
% 		\LARGE \textbf{\uppercase{what are you doing with the humans}}
% 		\HRule{2pt} \\ [0.5cm]
% 		\normalsize \today \vspace*{5\baselineskip}}

% \date{}

% \author{
% 		Rhys Gray }

% \maketitle
% \newpage

% %-------------------------------------------------------------------------------
% % Section title formatting
% \sectionfont{\scshape}
% %-------------------------------------------------------------------------------

% %-------------------------------------------------------------------------------
% % BODY
% %-------------------------------------------------------------------------------

% \section*{Program Notes}
% \emph{what are you doing with the humans} is a solo work for violin that explores half-harmonics.
% It is a non-programmatic work, and the title was inspired by a question that my supervisor posed to me while I sought ethics approval for my exegesis; a simple phrase laden with possible contexts, spurring the imagination to try and complete the meaning.



% It is, in a way, an etude that explores half-harmonics, similar to those found in Sciarrino's \emph{6 Caprricio for violin}. Half-harmonics are produced by applying left hand finger pressure halfway between that required to create a harmonic, and a \emph{normale} sound. The sound that is produced should be a mixture of the stopped string pitch, the harmonic pitch, and a resistant, slightly noisy quality.

% \section*{Notation}
% \begin{itemize}

%     \item Half-harmonics are notated in the score as a half-filled diamond notehead.
%     \item sp denotes \emph{sul ponticello}.
%     \item msp denotes \emph{molto sul ponticello}.
%     \item similarly, st denotes \emph{sul tasto}, and mst denotes \emph{molto sul tasto}
% \end{itemize}


% \newpage


% \title{ \normalsize \textsc{for solo violin}
% 		\\ [2.0cm]
% 		\HRule{0.5pt} \\
% 		\LARGE \textbf{\uppercase{Doppelganger}}
% 		\HRule{2pt} \\ [0.5cm]
% 		\normalsize \today \vspace*{5\baselineskip}}

% \date{}

% \author{
% 		Rhys Gray }

% \maketitle
% \newpage



% \section*{Program Notes}
% \emph{Doppelganger} is a piece for solo viola, written to explore the lower register of the viola using subharmonics juxtaposed with upper harmonics. 

% To play subharmonics, one should place the bow at the 6th partial of the harmonic series of the fingered pitch, and bow with excessive pressure and an absolutely consistent speed. 
% The increased pressure will distort the vibration of the string, producing a phase loop which, in turn, produces the subharmonic. 

% Subharmonics are achieved through precise control of torsional oscillation, which usually produces the sound of an amateur string player's heavy handed, slow bowing. 

% The production of subharmonics can be aided by using older strings (which work better due to fats building up on the strings). 
% Making a counter-clockwise half-twist in the string can also make it easier to produce octave and major second subharmonics (additional twists can help achieve lower subharmonics, at the expense of higher ones).

% \section*{Notation}
% \begin{itemize}

%     \item Subharmonics are notated in the score using a square notehead for the fingering, with a small notehead at the desired resultant pitch.
%     \item sp denotes \emph{sul ponticello}.
%     \item msp denotes \emph{molto sul ponticello}.
%     \item similarly, st denotes \emph{sul tasto}, and mst denotes \emph{molto sul tasto}
% \end{itemize}