% chapter5.tex (Chapter 5 of the thesis)

\chapter{Results}
This is where I will present my findings and draw conclusions on the various techniques that I explored through writing and workshopping in Chapter 2. This will be more broad, and I will make amendments to what I posited in Chapter 2. \lipsum[4]
\subsection{Background}
Provide a better understanding on the ways that these techniques can be incorporated idiomatically into a composer's practice.

\subsection{Research statement/problem}
There is a dearth of resources for composers interested in these techniques.

\subsection{Aim and scope of thesis}
Ways to incorporate and present these techniques in an idiomatic way that is intuitive.

\subsection{Significance of work}
The properties of these techniques and the ways that they can be used idiomatically.


\lipsum[4]

\section{Findings}
This is where I will be presenting my research as a manual for composers. It will follow the Dick model of categorization.\autocite{dickOtherFlute1989} It will also take into account how common the technique is, as well as notational challenges. \lipsum[5]

\section{Reflection}
\lipsum[4]

\lipsum[4]

\lipsum[4]