% chapter4.tex (Chapter 4 of the thesis)

\chapter{Findings}
This is where I will be presenting my research as a manual for composers. 
It will follow the Dick model of categorization.\autocite{dickOtherFlute1989} 
It will also take into account how common the technique is, as well as notational challenges.

\subsection{Subharmonics}
Subharmonics are a difficult technique, that lend themselves to solo works, or works where they can be brought to the forefront.
They are notably different to overpressure, but bleed over into non-pitched overpressure is common.

This, plus the difficulty in their execution, makes them unsuitable for melodic content.

Players may find that subharmonics are easier on older strings, and they may also find that adding twists to the string may also help, or hinder the production of subharmonics, referencing the table below. Composers seeking to make use of subharmonics extensively may wish to consider the below table.

\begin{table}[]
    \centering
    \resizebox{\textwidth}{!}{%
    \begin{tabular}{llllllll}
    \hline
    \multicolumn{1}{r}{\begin{tabular}[c]{@{}r@{}}number of twists/\\ subharmonic intervals\end{tabular}} & \multicolumn{1}{c}{1/2} & \multicolumn{1}{c}{1} & \multicolumn{1}{c}{2} & \multicolumn{1}{c}{3} & \multicolumn{1}{c}{4} & \multicolumn{1}{c}{5} & \multicolumn{1}{c}{6} \\ \hline
    minor 2nd                                                                                             & x                       & x                     &                       &                       &                       &                       &                       \\
    major 2nd                                                                                             & x                       & x                     &                       &                       &                       &                       &                       \\
    minor 3rd                                                                                             & x                       & x                     & x                     &                       &                       &                       &                       \\
    major 3rd                                                                                             & x                       & x                     & x                     & x                     &                       &                       &                       \\
    perfect 4th                                                                                           &                         &                       &                       & x                     & x                     &                       &                       \\
    dim. 5th                                                                                              &                         &                       &                       &                       & x                     & x                     &                       \\
    perfect 5th                                                                                           & x                       &                       &                       &                       &                       & x                     & x                     \\
    minor 6th                                                                                             &                         &                       &                       &                       &                       &                       & x                     \\
    octave                                                                                                & x                       & x                     & x                     & x                     & x                     &                       &                      
    \end{tabular}%
    }
    \end{table}

Composers looking to use this technique should be aware that it is not a standard technique, and instrumentalists will need copious amounts of practice and guidance in order to fully realise this technique.

\subsubsection{Notation of Subharmonics}
Subharmonics should be notated with a square notehead, and a small notehead (optionally in parenthesis) at the desired pitch.
The technique description and notation should be included in the performance notes.

\subsection{Works featuring subharmonics}
Gemini 
6 Caprices

\subsection{Multiphonics}
Multiphonics are easier to achieve on larger instruments, due to the need for precise ratio-based fingering to achieve the resonance of multiple partials.
The technique description and notation should be included in the performance notes.
\lipsum[4]

\subsubsection{Notation of Multiphonics}

\subsection{Works featuring multiphonics}
% TODO: Find multiphonic works - https://trello.com/c/o75LaLo8/13-find-multiphonic-works

Mari Kimura Six Caprices, No. 4 

Andrew Greenwald On Structure (2a) - for clarinet, violin, and cello

Stefano Scodanibbio composed e/statico - 1980

Håkon Thelin: oibbinadocS - 2004

Håkon Thelin: Glasperlenspiel - 2010

Michael Liebman: Sonata for double bass, 2.movement Legato sonore

% Iannis Xenakis' \emph{Theraps} for solo contrabass is \lipsum[1].\autocite[]{}

Kaija Saariaho Lichtbogen (1986)

Thrust (1989, rev. 1991),  Kimmo Hakola, Rubato (Adagio) 

Eivind Buene `Blacklight' (19)

% Brian Ferneyhough Trittico Per G.S

% Barry Guy Statements II

\subsection{Half-harmonics}

\subsubsection{Notation of Half-harmonics}
Half-harmonics 
The technique description and notation should be included in the performance notes.


\subsection{Works featuring half-harmonics}

Robert Rowe - Flood Gate (1989)

Salvatore Sciarriono - 6 Capricci for violin (1976) (no. 5)

Helmut Lachenmann, Gran Torso (mm. 1-7)

Trevor Bača - Al-Kitab Al-Khamr (2015)

Scherzo Alla Francescana (1990, revised 1994) by Claudio Pompili 

Mary Bellamy - Transference (?)

Sam Park - The Colour of Light (2010)

Jack Symmonds - Hell Is Murky (2018)

\section{Reflection}




\lipsum[4]