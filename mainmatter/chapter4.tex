% chapter4.tex (Chapter 4 of the thesis)

\chapter{Findings and Research Implications}
In this chapter, I will be presenting my findings as an instructive manual for composers and performers interested in using the techniques.
Because my folio of works has some degree of overlap in usage of techniques, this chapter will deal with each technique, rather than be a review of findings from each piece singularly.
Where relevant, I will include my findings from working with performers.

Due to the limited number of permutations of these techniques, the Dick model of categorization is unnecessary, as space is not at a premium.\autocite{dickOtherFlute1989} 
It will also take into account how common the technique is, as well as notational challenges.

\section{Subharmonics} \label{sec:subharmonics}
Subharmonics are a difficult technique that lend themselves to solo works, or works where they can be brought to the forefront.
They produce a sound lower than the fundamental through precise control of torsional oscillation, which usually produces the sound of an amateur string player's heavy handed, slow bowing. 
The timbre of overpressure can vary, but is identifiably pitched, and typically is somewhat nasal.
They are notably different to overpressure, but bleed over into non-pitched overpressure is common.
This, plus the difficulty in their execution, makes them unsuitable for melodic content.

Several different intervals are available as subharmonics, and a myriad of factors feed into which are easily replicable, but as a rule, octaves come most easily, with minor seconds, major sevenths, and perfect fifths coming after that.
Composers should be aware that producing the specific pitch is not \emph{necessarily} guaranteed.
Subharmonics are unable to be performed \emph{laissez vibrer}, as the pitch returns to the fundamental as soon as the bow is no longer in contact with the string.\autocite[]{appleseedFeedbackExploratorySession2019}
Due to the pressure needed, subharmonics are most comfortable at at least \emph{mezzo-forte}, although quieter subharmonics are possible. 
Sympathetic resonances are common at higher volumes.
Playing on the two inner strings is slightly harder due to the angle of attack being restricted to not inadvertently play double-stopped notes.



Composers looking to use this technique should be aware that it is not a standard technique, and instrumentalists will need copious amounts of practice and guidance in order to fully and reliably realise this technique.

\subsection{Considerations for players}
Players may find that subharmonics are easier on older strings, and they may also find that adding twists to the string may also help, or hinder the production of subharmonics, as shown in \autoref{tab:twistTable}.\autocite[]{kimuraHowProduceSubharmonics1999}
Botting notes that experimentations with octavic subharmonics yielded a pitch slightly flatter than an octave. He states \begin{quotation}
  `I developed a left hand finger technique whereby I rotate my hand slightly clockwise, pivoting on the finger stopping the string, which has the effect of sharpening the subharmonic enough to be more in tune with the fundamental.'\autocite[111]{bottingDevelopingPersonalVocabulary2019}
\end{quotation}

To `find' the subharmonic, an excellent method to practice is to play \emph{sul ponticello} at a \emph{forte} dynamic on an open string, and then move towards the fingerboard, keeping the pressure but slowing the bowing speed down.
Subharmonics are easier to find at nodal points, particularly the 6th node (closest to the bridge).\autocite[]{appleseedFeedbackSightreadingSession2019}

Bow pressure should be totally consistent throughout the bowing stroke; rather than the tapered start and finish of \emph{normale}, players should imagine a more binary stop and start, beginning and ending on the string.
The bow hair should remain flat throughout the stroke, in order to to exert the maximum amount of pressure.\autocite[]{kimuraHowProduceSubharmonics1999}
Different bow positions on the string can make different subharmonics speak more easily.\autocite[]{kimuraHowProduceSubharmonics1999}

\subsubsection{Instrument specific considerations for subharmonics}
Subharmonics are easier to produce on the contrabass with a lighter bow, preferably a cello bow, or alternatively a French style bow.\autocite[]{longSubharmonics2019}
Conversely, violin and violas may benefit from the use of a heavier bow.\autocite[]{appleseedFeedbackSightreadingSession2019}
The tension of the string appears to impact the feasibility enormously; 13 inch violas may have more success producing subharmonics on the A string than 18 inch violas.\autocite[]{appleseedFeedbackSightreadingSession2019}



\subsection{Notation of Subharmonics} \label{sec:notation-subharmonics}
Subharmonics should be notated with a square notehead, and a small notehead (optionally in parenthesis) at the desired pitch.
The technique description and notation should be included in the performance notes.
If the resultant pitch falls well outside the stave, a second stave can be used in the appropriate clef to ensure that the performer knows what pitch is intended, as seen in \autoref{fig:Excerpt from Risset's Variants}.\autocite[]{rissetVariants1995}
It should be noted that Risset's notation omits a fingered pitch, which is not recommended.

\subsection{Works featuring subharmonics }\label{sec:subharmonicsLiterature}

\begin{itemize}
    \item Mari Kimura - 6 Caprices for subharmonics for solo violin (1997) 
    \item Mari Kimura -Gemini for solo violin (1993)
    \item Mari Kimura - ALT in three movements for solo violin (1992)
    \item Mari Kimura - JanMaricana (for subharmonics) for solo violin (2016)
    \item Joshua Burel - Sonata No. 2 for violin and piano “Subharmonics”
    \item Jean-Claude Risset - Variants (1995)
    \item Robert Rowe - Submarine (1996)
\end{itemize}

\section{Multiphonics} \label{sec:multiphonics}
Multiphonics are easier to achieve on larger instruments, due to the need for precise ratio-based fingering to achieve the resonance of multiple partials.
The technique description and notation should be included in the performance notes.

\begin{quotation}
  `Multiphonics are notated as a harmonic position, with an `M' and the string number (I-IV). 
  The theoretical sounding pitches are given in a bracketed staff above the main stave.
  String multiphonics are achieved through clusters of close harmonic nodes, and by playing a harmonic close to the highest partial.
  Above the sounding pitches, the sounding partials are given (i.e. M IV [4th + 13th + 9th + 15th + 5th]).
  Note that not all of these pitches will actually sound in practice.
  The bow should exert slightly more pressure than usual and should be drawn with a consistent speed which should be slower than for harmonics.'
\end{quotation}

\subsubsection{Instrument specific considerations for multiphonics}
Multiphonics are easier on large instruments, as more precise pitching is possible with the longer strings.
Composers should be aware that of the pair, the multiphonic node closer to the bridge can be harder to produce because of this fact.

\subsection{Notation of Multiphonics} \label{sec:notation-multiphonics}
Much has been written about multiphonics, and they are a well established technique in woodwind writing.
The notation between them differs, though; precise fingering charts above resultant pitches do not translate precisely into string writing.
Fallowfield's method of denoting the multiphonic with a diamond notehead is, in the author's opinion, preferable to the alternatives.


\subsection{Works featuring multiphonics} \label{sec:multiphonicsLiterature}
% TODO: Find multiphonic works - https://trello.com/c/o75LaLo8/13-find-multiphonic-works

\begin{itemize}
    \item Mari Kimura - 6 Caprices for subharmonics for solo violin (1997) 
    \item Andrew Greenwald - On Structure (2a) - for clarinet, violin, and cello
    \item Stefano Scodanibbio - composed e/statico (1980)
    \item Håkon Thelin - oibbinadocS (2004)
    \item Håkon Thelin - Glasperlenspiel (2010)
    \item Michael Liebman - Sonata for double bass, 2. movement Legato sonore
    \item Kaija Saariaho - Lichtbogen (1986)
    \item Kimmo Hakola - Thrust, Rubato (1989, rev. 1991) 
    \item Eivind Buene - `Blacklight' (2019)
% Brian Ferneyhough Trittico Per G.S
% Iannis Xenakis' \emph{Theraps} for solo contrabass is \lipsum[1].\autocite[]{}
% Barry Guy Statements II
\end{itemize}

\section{Half-harmonics} \label{sec:half-harmonics}

\subsection{Notation of Half-harmonics} \label{sec:notation-half-harmonics}
Half-harmonics can be notated in one of several ways (see \autoref{fig:halfHarmonicNotationExamples}), but regardless of the chosen symbol, should be included and described in the performance notes.


\begin{figure}
    \includegraphics[width=\linewidth]{./resources/halfHarmonicNotationExamples.pdf}
    \caption{Half-harmonic notation examples} \label{fig:halfHarmonicNotationExamples}
  \end{figure}

It should be noted the first and last of the examples in \autoref{fig:halfHarmonicNotationExamples} do not have discrete noteheads for crotchets and minims like regular diamond noteheads.
As such, if there is rhythmic ambiguity, rhythms should be clarified above the stave as normal.
The first example, as seen in Sciarrino's \emph{Six Capricci for Violin} (\autoref{fig:sciarrinoExcerpt}) shows the ambiguity of this.
It should be noted that the `slant' of the regular harmonic notehead is opposite to the unfinished diamond, going from up to down.
This helps differentiate the two, but is perhaps not enough to make it easily distinguishable.

The second example unfortunately is not without issues, either; (\emph{normale}) harmonics denoted with a circle are exclusively for the resultant pitch.\autocite[419]{gouldBars2011} 
While half-harmonics \emph{do} produce the notated pitch, rapid transitions between half-harmonics and \emph{normale} harmonics using half-filled circles may cause confusion due to the translation between a symbol that denotes pressure needed and a resultant harmonic respectively, as illustrated in \autoref{fig:circleExample}
This is compounded by the circle notation's inability to handle harmonics that fall well outside the range of the staff (i.e. major 3rd and minor 3rd harmonics), resulting in a need for at least two types of notation; circular half-harmonics, and diamond noteheads for problematic \emph{normale} harmonics.

\begin{figure}
    % \includegraphics[width=\linewidth]{./resources/circleExample.pdf}
    \includegraphics[page=3,width=\textwidth]{resources/halfharmonicsExampleNotation.pdf}
    \caption{Half-harmonic circular notation} \label{fig:circleExample}
  \end{figure}

The third example of notation displayed in \autoref{fig:diamondSymbolNotation} is a non-standard symbol, and also suffers the same issues that plague the previous example.

\begin{figure}
  % \includegraphics[width=\linewidth]{./resources/circleExample.pdf}
  \includegraphics[page=4,width=\textwidth]{resources/halfharmonicsExampleNotation.pdf}
  \caption{Half-harmonic diamond symbol notation} \label{fig:diamondSymbolNotation}
\end{figure}

Appending text rather than a graphic may produce good results, as seen in \autoref{fig:textSymbolNotation}.

\begin{figure}
  % \includegraphics[width=\linewidth]{./resources/circleExample.pdf}
  \includegraphics[page=1,width=\textwidth]{resources/halfharmonicsExampleNotation.pdf}
  \caption{Half-harmonic displayed with text} \label{fig:textSymbolNotation}
\end{figure}

Compare this with \autoref{fig:halfFilledNotation} and \autoref{fig:halfEmptyNotation}, which is an example of Sciarrino's half-empty notation as seen in \autoref{fig:sciarrinoExcerpt}.

\begin{figure}
  % \includegraphics[width=\linewidth]{./resources/circleExample.pdf}
  \includegraphics[page=5,width=\textwidth]{resources/halfharmonicsExampleNotation.pdf}
  \caption{Half-harmonic half-filled notehead} \label{fig:halfFilledNotation}
\end{figure}


\begin{figure}
  % \includegraphics[width=\linewidth]{./resources/circleExample.pdf}
  \includegraphics[page=2,width=\textwidth]{resources/halfharmonicsExampleNotation.pdf}
  \caption{Half-harmonic half-empty notehead} \label{fig:halfEmptyNotation}
\end{figure}


  % \begin{figure}
  %   \includegraphics[width=\linewidth]{./resources/diamondExample.pdf}
  %   \caption{Half-harmonic diamond notation} \label{fig:diamondExample}
  % \end{figure}

  % \begin{figure}
  %   \includegraphics[width=\linewidth]{./resources/diamondExample2.pdf}
  %   \caption{Half-harmonic diamond notation with circle notation} \label{fig:diamondExample2}
  % \end{figure}

% Further simplications on the stave are possible as evidenced in \autoref{fig:halfHarmonicNotationExample3} by denoting the string using text, or if sequential, lines.

% \begin{figure}
%     \includegraphics[width=\linewidth]{./resources/halfHarmonicNotationExample3.pdf}
%     \caption{Half-harmonic diamond notation with string specification} \label{fig:halfHarmonicNotationExample3}
%   \end{figure}



It should be noted that the half-filled notehead as depicted in Gould and \autoref{fig:halfFilledNotation}, nor the Sciarrino style half-empty notehead as seen \autoref{fig:halfEmptyNotation} are not available in modern versions of Sibelius or Dorico as of the time of writing.\autocite[424]{gouldBars2011}
The flagship Standard Music Layout Font (SMuFL), Bravura, includes the half-harmonic circle as depicted in \autoref{fig:circleExample}, but is only available on Dorico and the Sibelius port of Bravura, Norfolk.\autocite[]{w3ccommitteeStandardMusicFont2019}

\subsection{Works featuring half-harmonics} \label{sec:half-harmonicsLiterature}

\begin{itemize}
    \item Robert Rowe - Flood Gate (1989)
    \item Salvatore Sciarriono - 6 Capricci for violin (no. 5) (1976) 
    \item Helmut Lachenmann, Gran Torso
    \item Trevor Bača - Al-Kitab Al-Khamr (2015)
    \item Claudio Pompili - Scherzo Alla Francescana (1990, revised 1994)
    \item Mary Bellamy - Transference (?)
    \item Sam Park - The Colour of Light (2010)
    \item Jack Symmonds - Hell Is Murky (2018)
\end{itemize}

\section{Reflection}




\lipsum[4]