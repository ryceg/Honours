% chapter4.tex (Chapter 4 of the thesis)

\chapter{Findings}
This is where I will be presenting my research as a manual for composers. 
It will follow the Dick model of categorization.\autocite{dickOtherFlute1989} 
It will also take into account how common the technique is, as well as notational challenges.

\subsection{Subharmonics}
Subharmonics are a difficult technique, that lend themselves to solo works, or works where they can be brought to the forefront.
They are notably different to overpressure, but bleed over into non-pitched overpressure is common.
This, plus the difficulty in their execution, makes them unsuitable for melodic content.
Composers looking to use this technique should be aware that it is not a standard technique, and instrumentalists will need copious amounts of practice and guidance in order to fully realise this technique.

\subsubsection{Notation}
Subharmonics should be notated with a square notehead, and a small notehead (optionally in parenthesis) at the desired pitch.

\subsection{Multiphonics}
\lipsum[4]

\subsubsection{Notation}
\lipsum[5]

\section{Reflection}




\lipsum[4]