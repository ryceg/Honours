% chapter4.tex (Chapter 4 of the thesis)

\chapter{Compositions}
My folio of works comprises of four pieces. `Doppelganger', for solo viola, `The Veldt', for solo contrabass and electronics, `Placeholder', for solo cello, and `SecondPlaceholder', for string quartet.
\subsection{Background}
Implement the experiments in a musical context.
\subsection{Research statement/problem}
Compositions will show both how these techniques can be used idiomatically, and how they can inform my craft.
\subsection{Aim and scope of thesis}
Writing works which will increase the collective understanding of how to implement these techniques.
\subsection{Significance of work}
Incorporating these techniques into my compositional process will show the pitfalls and ways that these techniques can be used.
\section{`Doppelganger'}
% TODO: Write Doppelganger
\emph{Doppelganger} is a piece for solo viola, written to explore the lower register of the viola using subharmonics juxtaposed with upper harmonics. 

\subsection{Findings of `Doppelganger'}
Workshopping `Doppelganger' \lipsum[3]

\section{`The Veldt'}
% TODO: Write The Veldt
Inspired by the eponymous short story by Ray Bradbury, \textit{The Veldt} is a composition for solo contrabass with electronics. Similarly like the namesake, this world is filled with danger but also beauty. It is non-programmatic, and my intent with Veldt was to create a soundworld and space that the performer was able to `roam around' in, and features several sections of improvisation on pitch-sets. \lipsum[1]

\subsection{Findings of `The Veldt'}
\lipsum[3]