% chapter4.tex (Chapter 4 of the thesis)

\chapter{Findings}
This is where I will be presenting my research as a manual for composers. 
Due to the limited number of permutations of these techniques, the Dick model of categorization is unnecessary, as space is not at a premium.\autocite{dickOtherFlute1989} 
It will also take into account how common the technique is, as well as notational challenges.

\subsection{Subharmonics} 
Subharmonics are a difficult technique, that lend themselves to solo works, or works where they can be brought to the forefront.
They are notably different to overpressure, but bleed over into non-pitched overpressure is common.

This, plus the difficulty in their execution, makes them unsuitable for melodic content.

Players may find that subharmonics are easier on older strings, and they may also find that adding twists to the string may also help, or hinder the production of subharmonics, referencing the table below. 
Composers seeking to make use of subharmonics extensively may wish to consider the following table.

\begin{table}
    \centering
    \caption{Relation between twists in string and resultant subharmonics}
    \begin{tabular}{llllllll} 
    \toprule
    \multicolumn{1}{r}{} & \multicolumn{1}{c}{1/2} & \multicolumn{1}{c}{1} & \multicolumn{1}{c}{2} & \multicolumn{1}{c}{3} & \multicolumn{1}{c}{4} & \multicolumn{1}{c}{5} & \multicolumn{1}{c}{6}  \\ 
    \hline
    minor 2nd            & x                       & x                     &                       &                       &                       &                       &                        \\
    major 2nd            & x                       & x                     &                       &                       &                       &                       &                        \\
    minor 3rd            & x                       & x                     & x                     &                       &                       &                       &                        \\
    major 3rd            & x                       & x                     & x                     & x                     &                       &                       &                        \\
    perfect 4th          &                         &                       &                       & x                     & x                     &                       &                        \\
    diminished 5th       &                         &                       &                       &                       & x                     & x                     &                        \\
    perfect 5th          & x                       &                       &                       &                       &                       & x                     & x                      \\
    minor 6th            &                         &                       &                       &                       &                       &                       & x                      \\
    octave               & x                       & x                     & x                     & x                     & x                     &                       &                        \\
    \bottomrule
    \end{tabular}
    \end{table}\autocite[]{kimuraHowProduceSubharmonics1999}

    Curiously, older strings work better for production of subharmonics due to the fats that accumulate on the string, and the lower strings are more suitable due to the pressure needed.\autocite{kimuraHowProduceSubharmonics1999}
Composers looking to use this technique should be aware that it is not a standard technique, and instrumentalists will need copious amounts of practice and guidance in order to fully realise this technique.

\subsubsection{Notation of Subharmonics}
Subharmonics should be notated with a square notehead, and a small notehead (optionally in parenthesis) at the desired pitch.
The technique description and notation should be included in the performance notes.

\subsection{Works featuring subharmonics}

\begin{itemize}
    \item 6 Caprices for subharmonics for solo violin, '97.
    \item Gemini for solo violin, '93.
    \item ALT in three movements for violin solo, '92.
    \item Sonata No. 2 for violin and piano “Subharmonics” - Joshua Burel
    \item Jean-Claude Risset - Variants
    \item Robert Rowe - Submarine
\end{itemize}

\subsection{Multiphonics}
Multiphonics are easier to achieve on larger instruments, due to the need for precise ratio-based fingering to achieve the resonance of multiple partials.
The technique description and notation should be included in the performance notes.
\lipsum[4]

\subsubsection{Notation of Multiphonics}

\subsection{Works featuring multiphonics}
% TODO: Find multiphonic works - https://trello.com/c/o75LaLo8/13-find-multiphonic-works

\begin{itemize}
    \item Mari Kimura Six Caprices, No. 4 
    \item Andrew Greenwald On Structure (2a) - for clarinet, violin, and cello
    \item Stefano Scodanibbio composed e/statico - 1980
    \item Håkon Thelin: oibbinadocS - 2004
    \item Håkon Thelin: Glasperlenspiel - 2010
    \item Michael Liebman: Sonata for double bass, 2.movement Legato sonore
    \item Kaija Saariaho Lichtbogen (1986)
    \item Thrust (1989, rev. 1991),  Kimmo Hakola, Rubato (Adagio) 
    \item Eivind Buene `Blacklight' (19)
% Brian Ferneyhough Trittico Per G.S
% Iannis Xenakis' \emph{Theraps} for solo contrabass is \lipsum[1].\autocite[]{}
% Barry Guy Statements II
\end{itemize}

\subsection{Half-harmonics}

\subsubsection{Notation of Half-harmonics}

The technique description and notation should be included in the performance notes.


\subsection{Works featuring half-harmonics}

\begin{itemize}
    \item Robert Rowe - Flood Gate (1989)
    \item Salvatore Sciarriono - 6 Capricci for violin (1976) (no. 5)
    \item Helmut Lachenmann, Gran Torso
    \item Trevor Bača - Al-Kitab Al-Khamr (2015)
    \item Scherzo Alla Francescana (1990, revised 1994) by Claudio Pompili 
    \item Mary Bellamy - Transference (?)
    \item Sam Park - The Colour of Light (2010)
    \item Jack Symmonds - Hell Is Murky (2018)
\end{itemize}

\section{Reflection}




\lipsum[4]