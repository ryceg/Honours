% chapter2.tex (Chapter 2 of the thesis)

\chapter{Assessment of Harmonic Based Techniques and Repertoire}

% //TODO: Talk about Pythagoras - https://trello.com/c/uo0AALKP/1-talk-about-pythagoras
Goals for this chapter:

1. Explain sound production of stringed instruments.
2. Explain the way in which the techniques differ from standard sound production.
3. Explain the qualities of the techniques.
4. Explain the notation.



Harmonic based techniques invariably make use of the harmonic series in one way or another. 
The harmonic series is a sequence of tones in which the frequency of each is an integer multiple of the fundamental frequency. 
The earliest forms of tuning systems were based around these, but modern instruments are tuned using equal temperament. 
The pitch of sound on stringed instruments is determined via tension, effecting the speed (and consequently pitch) the string vibrates at. 
Altering the tension is most commonly achieved via fingerings on the instrument's fingerboard, but bow pressure can also play a part in pitch production (see subharmonics).

The objective categorisation of techniques is a Sisyphean task due to the variability of the techniques, but general guidelines can be made; Dick's \emph{The Other Flute} makes good use of quantifying qualitative data about the properties of multiphonics, and the idea of his tables will be used, adapting the format to each technique.\autocite[84]{dickOtherFlute1989}

To be able to pass any judgement on the techniques, we must first understand these techniques' capabilities, limitations, qualities, considerations, and values. 
Without references to other composers' works, any implication of authority on what constitutes as `idiomatic' writing is baseless. 
As such, references to other works will be used to support claims. 
Where no such references are available, it will be marked as the author's personal opinion. 
Even without any available references to substantiate compositions as idiomatic, their creation contributes to the literature, and thus can be used if not as an example, a warning on what not to do. 

\subsection{Background}
All of the techniques covered in this exegesis involve the excitation of a string instrument's string in a non-standard way. 
A small amount of understanding the physics behind these techniques is required, though they are not fully understood.
Strings create sound via the Helmholtz motion, which 

Subharmonics are perhaps a misnomer, and do not \emph{technically} fall under the branch of harmonic based techniques.
This is because their production is not by ways of the overtone (or undertone) series, as the pitches that can be produced do not follow any discernable ratio based pattern.



% Provide an overlay of the techniques and explain how they work, the general benefits and such.
% \subsection{Research statement/problem}
% Techniques are under-developed and/or under-used.

% \subsection{Aim and scope of thesis}
% Examples of use in current literature will support use-case scenarios, dearths of usage will support the fact that they are underused.

% \subsection{Significance of work}
% The production of technique and its uses.
\newpage
\section{Subharmonics} \label{sec:subharmonicsDiscussion}
% TODO: Explain subharmonics - https://trello.com/c/HP0b1P3h/2-explain-subharmonics
First discovered by Mari Kimura, subharmonics are a type of overpressure which produces a sound lower than the fundamental.\autocite{kimuraHowProduceSubharmonics1999} 
When the bow is drawn across the string, the drag of the bow twists the string, creating torsional oscillation. 
Under the right conditions, these can interact with the string to produce an identifiable pitch lower than the fundamental.\autocite{Subharmonics2006} 
One of the newest string techniques, subharmonics are still in their comparative infancy, and their notation has not been formalised. 

Subharmonics represent an incredible opportunity for solo string repertoire. 
On higher pitched instruments, their use can provide harmonic support (particularly in cadenza passages) and extend the range of the instrument. 
On lower pitched instruments, subharmonics function better as a timbral mechanic, much like overpressure. 




\subsection{Subharmonics in the literature}

Subharmonics share a common relative with woodwind and brass instrument in pedal tones.
They operate in similar ways, though the method of production differs greatly, and the reliability of production of subharmonics is much lower than the equivalent on brass instruments.
Because of these commonalities, it is not unreasonable to make comparisons between subharmonics and its equivalents, especially in regards to notation and implementation in works.

There have been several different ways of notating them, each with their advantages and disadvantages.

Possibly the first person to make use of the technique, Crumb described what we know as subharmonics as `pedal tones'.\autocite{crumbBlackAngelsImages1971} 
The use of square noteheads and a separate stave for the resultant pitch makes the technique clear and readily understandable.\autocite[]{crumbBlackAngels1995}
\begin{figure}
    \includegraphics[width=\linewidth]{./resources/crumbBlackAngels.png}
    \caption{Excerpt from Crumb's \emph{Black Angels}}
    \label{fig:Excerpt from Crumb's Black Angels}
\end{figure}
% TODO: Citation is needed for Crumb - https://trello.com/c/Rpypkzbm/4-citation-needed-for-crumb
% TODO: Citation needed for pedal tones - https://trello.com/c/03arTJkS/5-citation-needed-for-pedal-tones



Gerard Grisey's \emph{Vortex Temporum} features overpressure, with a subharmonic of specifically a seventh.\autocite[]{griseyVortexTemporum}
He notates the subharmonic technique using a triangular filled notehead showing the intended pitch, along with a double down-bow, with an arrow above it, shown in \autoref{fig:Excerpt from Grisey's playing instructions for Vortex Temporum}. 
Somewhat abstracted out, this hides the intended effect behind symbols, and is slower to sight read.


\begin{figure}
  \includegraphics[width=\linewidth]{./resources/griseyVortexTemporum.jpg}
  \caption{Excerpt from Grisey's playing instructions for \emph{Vortex Temporum}.}
\label{fig:Excerpt from Grisey's playing instructions for Vortex Temporum}
\end{figure}


Mari Kimura's \emph{Gemini} (\autoref{fig:Excerpt from Kimura's Gemini}) is an example of idiomatic usage of subharmonics on the violin.\autocite[]{kimuraGemini1992}
Kimura's notation practice of using a harmonic denoting the intended pitch below the fundamental is similar to the standard notation of harmonics, which Gould states is to `write harmonics as the player will finger them.'\autocite[413]{gouldBars2011} 
Unfortunately, this method proved somewhat counterintuitive in practice, as the notation was too similar, and caused sight reading issues.

  
\begin{figure}
  \includegraphics[width=\linewidth]{./resources/kimura_gemini.png}
  \caption{Excerpt from Kimura's \emph{Gemini}}
\label{fig:Excerpt from Kimura's Gemini}
\end{figure}
% TODO: Citation needed for Gemini

  Botting notes that experimentations with octavic subharmonics yielded a pitch slightly flatter than an octave. He states \begin{quotation}
    `I developed a left hand finger technique whereby I rotate my hand slightly clockwise, pivoting on the finger stopping the string, which has the effect of sharpening the subharmonic enough to be more in tune with the fundamental.'\autocite[111]{bottingDevelopingPersonalVocabulary2019}
\end{quotation}

Players may find that subharmonics are easier on older strings, and they may also find that adding twists to the string may also help, or hinder the production of subharmonics, as shown in \autoref{tab:twistTable}. 

\begin{table}
  \centering
  \caption{Relation between twists in string and resultant subharmonics}
  \label{tab:twistTable}
  \begin{tabular}{llllllll} 
  \toprule
  \multicolumn{1}{r}{} & \multicolumn{1}{c}{1/2} & \multicolumn{1}{c}{1} & \multicolumn{1}{c}{2} & \multicolumn{1}{c}{3} & \multicolumn{1}{c}{4} & \multicolumn{1}{c}{5} & \multicolumn{1}{c}{6}  \\ 
  \hline
  minor 2nd            & x                       & x                     &                       &                       &                       &                       &                        \\
  major 2nd            & x                       & x                     &                       &                       &                       &                       &                        \\
  minor 3rd            & x                       & x                     & x                     &                       &                       &                       &                        \\
  major 3rd            & x                       & x                     & x                     & x                     &                       &                       &                        \\
  perfect 4th          &                         &                       &                       & x                     & x                     &                       &                        \\
  diminished 5th       &                         &                       &                       &                       & x                     & x                     &                        \\
  perfect 5th          & x                       &                       &                       &                       &                       & x                     & x                      \\
  minor 6th            &                         &                       &                       &                       &                       &                       & x                      \\
  octave               & x                       & x                     & x                     & x                     & x                     &                       &                        \\
  \bottomrule
  \end{tabular}
  \end{table}\autocite[]{kimuraHowProduceSubharmonics1999}


\subsection{Notation of Subharmonics}

The example used on Long's website, \emph{The Modern Double Bass} (\autoref{fig:Notation of subharmonics from Long's website, The Modern Double Bass}) features a square notehead with the intended sound at pitch in a bracketed notehead, with harmonics and a technique line of `S.H'.\autocite[]{longSubharmonics2019}

\begin{figure}
  \includegraphics[width=\linewidth]{./resources/longSubharmonicNotation.jpg}
  \caption{Notation of subharmonics from Long's website, The Modern Double Bass}\autocite[]{longSubharmonics2019}
\label{fig:Notation of subharmonics from Long's website, The Modern Double Bass}
\end{figure}

It is the author's opinion that this is somewhat redundant, as just square noteheads with the intended produced pitch would be enough to delineate the technique. 
The technique line is supernumerary, and it would only be advisable to use it in extended passages of uninterrupted subharmonics.

Notationally, the best practice appears to be following Crumb's approach, condensing into one stave where possible. 

% TODO: reference risset and rowe - https://trello.com/c/wDhTSCzs/29-reference-risset-and-rowe

Jean-Claude Risset's \emph{Variants}, written for Kimura, is a work that makes use of both subharmonics and digital processing of live sound. 
It uses a separate stave for the subharmonics and digital processing, as seen in \autoref{fig:Excerpt from Risset's Variants}. 

\begin{figure}
  \includegraphics[width=\linewidth]{./resources/rissetALFExcerpt.pdf}
  \caption{Excerpt from Risset's \emph{Variants}}
\label{fig:Excerpt from Risset's Variants}
\end{figure}

Rowe also wrote a work for Kimura, and uses a regular notehead for fingering, with a square bracketed cue sized notehead, as seen in \autoref{fig:Excerpt from Rowe's Submarine}.
This combines the best of both worlds, keeping the score free from clutter when not needed.
It should be noted that Rowe uses notation that has fallen out of style to notate \emph{normale} harmonics, notating the fingering position with the diamond notehead, and the resultant pitch with a harmonic circle above it. 
Harmonic circles always denote resultant pitch, and the inclusion of the fingering obfuscates this.\autocite[420]{gouldBars2011}


\begin{figure}
  \includegraphics[width=\linewidth]{./resources/roweALFExcerpt.pdf}
  \caption{Excerpt from Rowe's \emph{Submarine}}
\label{fig:Excerpt from Rowe's Submarine}
\end{figure}


Musicians are better at sight reading above the stave than below the stave, so unlike natural harmonics, the need to split into another stave to show the resultant pitch is likely to be more common for composers wishing to use subharmonics. 
Subharmonics are explored in my works \nameref{sec:bassPiece}, and \nameref{sec:violaPiece}.


% TODO: undefined - https://trello.com/c/SU5ZEEmJ/6-- Do I need a citation for "musicians are better at sight reading above the stave than below"?
\newpage
\section{Multiphonics} \label{sec:multiphonicsDiscussion}
% TODO: Explain multiphonics - https://trello.com/c/6vkcY6CO/7-explain-multiphonics

Multiphonics are most commonly the domain of wind, and occasionally brass instruments, but they are an emerging technique in string writing. 
They are produced when fingerings split the string between two natural harmonics, allowing for the string to resonate at multiple frequencies.
% TODO: Citation needed for explanation of physics of multiphonics - https://trello.com/c/EboMHDaN/8-citation-needed-for-explanation-of-physics-of-multiphonics
Multiphonics on stringed instruments are difficult, but with appropriate preparation and notation, are feasible. 
Production of multiphonics, as with wind instruments, is not guaranteed, and can be dependant on a variety of external factors, including the humidity, make of the instrument, bow used, and other variables that are outside of the control of a composer. 
% TODO: Citation needed for the factors leading to multiphonics - https://trello.com/c/9iC7EAXb/24-citation-needed-for-the-factors-leading-to-multiphonics

Multiphonics are fragile, and require much preparation to execute reliably. 
Despite this, they can be used to achieve harmonies that are not otherwise achieveable through double-stopping, and lend themselves well to drawn out or slow passages of music. 
Multiphonics' exact pitching makes them ideal for music that uses ratios, microtones, or tone rows. 

\subsection{Multiphonics in the literature}

Fallowfield explores multiphonic production on the cello in her thesis CelloMap comprehensively, with video recordings of all possible multiphonics and permutations, including pizzicati.\autocite{fallowfieldCelloMapHandbook2009} 
These are isolated, though, and give little indication to the difficulty of the multiphonics.

Ashley John Long's `The Modern Double Bass' website serves a similar purpose as Fallowfield's CelloMap for the double bass\autocite{longModernDoubleBass}. 
He divides them into different categories as detailed below, some of which have more information and detail than others. 
Despite the varying degrees of detail, his work on cataloguing multiphonics is more in depth than many other resources.

\begin{table}[]
  \centering
  \resizebox{\textwidth}{!}{%
  \begin{tabular}{@{}ll@{}}
  \toprule
  \textbf{Type}                                      & \textbf{Description}                                                       \\ \midrule
  `Natural' multiphonics                             & Chart of different fingerings, similar to Fallowfield.                     \\ \midrule
  Pizzicato multiphonics                             & Description of technique, production, and result.                          \\ \midrule
  Textural multiphonics                              & Description of technique, production, result, and considerations.          \\ \midrule
  Multiphonics behind the bridge                     & Description of technique.                                                  \\ \midrule
  Artificial multiphonics                            & Chart of different fingerings, similar to Fallowfield.                     \\ \midrule
  Percussive multiphonics                            & Description of technique, production, result, and considerations.          \\ \midrule
  Timbral multiphonics                               & Description of technique.                                                  \\ \midrule
  Transformative multiphonics                        & Description and production of technique                                    \\ \midrule
  Multiphonics through Variations in Finger Pressure & Description of technique, production, result, considerations, and example. \\ \bottomrule
  \end{tabular}%
  }
  \end{table}




\subsection{Notation of Multiphonics}

It should be noted that multiphonics are markedly different to the multiphonics of wind instruments due to the fingering systems.
While wind instruments achieve multiple tones by exploiting the construction of their instrument, each string multiphonic is produced agnostic of fingerings.
As such, the challenges that string multiphonic notation face are different to wind instruments.
With no fingering chart necessary, string instrument multiphonics also have no frame of reference for what sounds can be expected to be produced.
String instruments also are not solely monophonic instruments, so notating the resultant multiphonic on the stave produces confusing results.
Therefore, another system of denoting multiphonics must be used, as the existing wind literature is not suited for the purpose.

Buene uses a chart of diamond noteheads with their corresponding intended multiphonic in the score for his work for two double basses, \emph{Blacklight}.\autocite[39-42]{thelinMultiphonicsDoubleBass2011}
It mimics Fallowfield's charts of corresponding nearby quartertones, though the diamond notehead is already used for harmonics, which has the potential to cause confusion.
\begin{figure}
  \includegraphics[width=\linewidth]{./resources/bueneMultiphonicNotation.png}
  \caption{Excerpt from Buene's Blacklight.}
\label{fig:Excerpt from Buene's Blacklight}
\end{figure}

Thelin's thesis on double bass multiphonics states:
\begin{quotation}
    `Multiphonics is [sic] always notated with the harmonic diamond sign, in tablature notation
indicating finger positions rather than musical pitches. I suggest using the symbol M. above or
below the note to indicate that it is a multiphonic sound, together with the indication on which
string to play the note (in Roman numerals).'\autocite[6]{thelinMultiphonicsDoubleBass2011}
\end{quotation}

\begin{figure}
    \includegraphics[width=\linewidth]{./resources/thelinMultiphonicNotation.png}
    % TODO: What is an excerpt from a thesis called? - https://trello.com/c/6xHaco1o/10-what-is-an-excerpt-from-a-thesis-called
    \caption{Excerpt from Thelin's thesis.}
  \label{fig:Excerpt from Thelin's thesis}
  \end{figure}
%   TODO: Reword Fallowfield dual harmonic positions
His notation suggestion is a somewhat less sophisticated version of Fallowfield's suggestion to notate the approximate pitch down to the cent necessary to produce the multiphonic. 
Due to the symmetry of the production of harmonics on the string, Fallowfield specifies both upper and lower positions necessary to produce the same multiphonic.\autocite[index/the-string/multiphonics-and-other-multiple-sounds/fingeringcharts.html]{fallowfieldCelloMap}
\begin{figure}
    \includegraphics[width=\linewidth]{./resources/fallowfieldMultiphonicFingering.png}
    % TODO: What is an excerpt from a website called? - https://trello.com/c/04ESsMtJ/9-what-is-an-excerpt-from-a-website-called
    \caption{Excerpt from Fallowfields's website.}\autocite[]{fallowfieldCelloMap}
\label{fig:Excerpt from Fallowfields's website}
  \end{figure}

  We can see this in practice in Oliver Thurley's work for solo contrabass, \emph{yet another example of the porousness of certain borders}, where he adds another stave showing the intended pitches to be produced.\autocite{thurleyAnotherExamplePorousness2014}

  \begin{figure}
    \includegraphics[width=\linewidth]{./resources/thurleyMultiphonicNotation.png}
    % TODO: What is an excerpt from a piece called? - https://trello.com/c/E9QqxFt0/12-what-is-an-excerpt-from-a-piece-called
    % TODO: Quotation marks in figure labels? - https://trello.com/c/HLJwnAwa/11-quotation-marks-in-figure-labels
    \caption{Excerpt from Thurley's \emph{yet another example of the porousness of certain borders}\autocite[]{thurleyAnotherExamplePorousness2014}}
\label{fig:Excerpt from Thurley's `yet another example of the porousness of certain borders'}
  \end{figure}

Thurley embraces the fragility of these multiphonics, and uses their variability as a feature, rather than a hindrance. 
Slow, quiet transitions between multiphonics, double-stopped harmonics, and other extended techniques make the occasional unintentional destabilisation of a multiphonic a point of textural interest.

\newpage
\section{Half-Harmonics} \label{sec:halfHarmonicsDiscussion}
% TODO: Explain half-harmonics - https://trello.com/c/0v3lKvmZ/25-explain-half-harmonics
Half-harmonics is a term assigned to the fingering pressure found somewhere in between a regular note and harmonic. 
The technique is not difficult to produce, and the resultant sound is not dissimilar to the fragility of a multiphonic, producing both the fundamental pitch, and the harmonic. 
It should be noted that the half-harmonic is a modifying left-hand technique; it can be applied to multiphonics (although the resultant sound would likely be more noise than discernably either of the two techniques), but is not compatible with subharmonics due to the bow pressure needed to produce subharmonics eliminating the possibility of half-harmonics being produced.
The terminology has not been formalised, but is most widely known as half-harmonics, although some works describe the technique without ascribing a name.

\subsection{Half-harmonics in the literature}
Half-harmonics, like the other techniques covered in this exegesis, have relatives in the wind and brass literature. 
Half-fingered and half-valved techniques appear in the respective nomenclature, and share common attributes of speaking poorly with bleedover into partials with the half-harmonic technique.
Unlike multiphonics, the mechanical production of the technique is not dissimilar to the wind and brass facsimiles; all three families' respective techniques revolve around pressing almost to the point of a \emph{normale} sound, but not quite, resulting in a pinched sound.
Because of this, it is not unreasonable to draw parallels between the half-valved and fingered literature, and half-harmonic literature. 

Half-harmonics do not feature heavily in the literature, with the most notable work being Sciarrino's 6 Caprices for solo violin.\autocite{sciarrinoCapricciViolino1976} 
Lachenmann also makes use of them, and states it is
\begin{quotation}
  [\dots] `important not to produce any harmonics here; the result should be a veiled, almost immaterial and hardly perceptible coloring of the dominating string sound produced by the stopped note'\autocite[foreword]{lachenmannMusikFurStreichquartett1972}
\end{quotation}
% Half-harmonics are used typically as a colourant, rather than a feature technique, and 

\begin{figure}
  \includegraphics[width=\linewidth]{./resources/sciarrinoHalfHarmonicNotation.pdf}
  \caption{Excerpt from Sciarrino's 5th capriccio from \emph{6 Capricci for Violin}}
\label{fig:sciarrinoExcerpt}
\end{figure}


\subsection{Notation of half-harmonics}
Perhaps the most straight-forward technique covered in this exegesis, notation for half-harmonics have just a single variable of finger pressure to convey in notation.
The use of standard notation, modified to reflect the idea that the technique fits in 'half way between' two well established techniques (normale and harmonics) would be ideal, conforming to Gould's ideology of maintaining uniformity.



% TODO: Add Gould reference for creating new notation - https://trello.com/c/pAR2S6wg/26-add-gould-reference-for-creating-new-notation