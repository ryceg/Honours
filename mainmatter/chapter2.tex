% chapter2.tex (Chapter 2 of the thesis)

\chapter[Assessment of Harmonic Based Techniques]{Assessment of Harmonic Based Techniques}

Harmonic based techniques invariably make use of the harmonic series in one way or another. The harmonic series is a sequence of tones in which the frequency of each is an integer multiple of the fundamental frequency. The earliest forms of tuning systems were based around these, but modern instruments are tuned around equal temperament.

\subsection{Subharmonics}
% TODO: Explain subharmonics
First discovered by Mari Kimura, subharmonics work via a \lipsum[1]\autocite{kimuraHowProduceSubharmonics1999} 

\subsection{Multiphonics}
% TODO: Explain multiphonics
Multiphonics on stringed instruments are difficult, but with appropriate preparation and notation, are quite feasible. Production of multiphonics, as with wind instruments, is not guaranteed, and can be dependant on a variety of external factors, including the humidity, make of the instrument, bow used, and other variables that are outside of the control of a composer.

% TODO: Expand Fallowfield discussion.
Fallowfield explores multiphonic production on the cello in her thesis CelloMap.\autocite{fallowfieldCelloMapHandbook2009} \lipsum[1]

\subsection{Electronically-Assisted Harmonics}
% TODO: Expand electronic assisted harmonics.
The use of electronics to augment acoustic instruments is hardly a new technique, although their use in a live context is still relatively inaccessible. For my composition `Veldt', I will be making use of Max MSP, a patch-based audiovisual processing application. 

\newpage
\section{`Doppelganger'}
% TODO: Write Doppelganger
\textit{Doppelganger} is a piece for solo viola, written to explore the lower register of the viola using subharmonics juxtaposed with upper harmonics. Kimura's notation practice of using a harmonic denoting the intended pitch below the fundamental is similar to the standard notation of harmonics. Gould states that the standard approach is to `write harmonics as the player will finger them.\autocite[413]{gouldBars2011}' This is a reasonable method, and rehearsals with violists proved successful. 

\subsection{Findings of `Doppelganger'}
Workshopping `Doppelganger' \lipsum[3]

\section{`The Veldt'}
% TODO: Write The Veldt
Inspired by the eponymous short story by Ray Bradbury, \textit{The Veldt} is a composition for solo double bass with electronics. Similarly like the namesake, this world is filled with danger but also beauty. It is non-programmatic, and my intent with Veldt was to create a soundworld and space that the performer was able to `roam around' in, and features several sections of improvisation on pitch-sets. \lipsum[1]

\subsection{Findings of `The Veldt'}
\lipsum[3]

