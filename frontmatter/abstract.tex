\chapter*{Abstract}
 
% \UseFirstVersion{\textit{This bit of text is in \/}\textsf{abstract.tex}. \textit{The
%  macros\/} \backslash{}\texttt{UseFirstVersion},
%  \backslash{}\texttt{UseSecondVersion}, \textit{and\/}
%  \backslash{}\texttt{UseBothVersions} \textit{are defined in
%  main.tex.} \textbf{Change ``First'' above to ``Second'' to make
%  this text go away.{}}

% I propose to explore a range of extended techniques that utilise the harmonic series and assess how they can be used in my, and other people's, creative practice. 
% For the scope of this exegesis, these extended techniques have been limited to string instruments, and include multiphonics, subharmonics, and half-harmonics. 
This exegesis explores compositional applications of the extended string techniques half-harmonics, subharmonics, and multiphonics.
% The examples and discussion in the exegesis typically make use of stringed instruments, since strings are the most prevalent in the literature and in repertoire. 

% While some techniques such as harmonics are well established and understood, others, such as subharmonics, are still immature in terms of both repertoire and resources available. 
% The timbral potentials of these techniques are uncharted territories and collectively represent a whole sound world that remains relatively inaccessible to contemporary art music composers.
A review of the literature and resources that are readily available to composers will be made to assess what techniques require further investigation and refinement. 
% I will conduct a review of the literature and resources that are readily available to composers to assess what techniques require further investigation and refinement. 
By researching these techniques and the mechanics behind them, using document analysis, and analysing recordings made, a better understanding of how these techniques can be implemented in my practice will form. 
As part of both the analysis of techniques and my compositional practice, I assess not only the compositional potential, but also the practicality of techniques. 
Reviewing the feasibility and notational aspects of the techniques will render the exegesis a practical document to reference for performance and composition. 
% I aim for my resulting exegesis to become a useful reference source for artists interested in learning about the mechanics, qualities, and potential implementations of these harmonic based extended techniques. 
The works that I compose accompanying the exegesis will show idiomatic treatment of the techniques and serve as references as such in the exegesis. 
The dissemination of the material I research will contribute to the accessibility of new sound possibilities for artists. 